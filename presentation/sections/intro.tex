\section{Introduction}

\begin{frame}
	\frametitle{What is a creative idea?}
	\begin{itemize}
		\item Any idea that:
		\begin{itemize}
			\pause
			\item is novel
			\pause
			\item has value (e.g. monetary, aesthetic)
		\end{itemize}
	\end{itemize}
\end{frame}

\begin{frame}
	\frametitle{What is a creative story?}
	\begin{itemize}
		\item Any story that:
		\begin{itemize}
			\pause
			\item is novel
			\pause
			\item has value (is still recognizable as a story)
		\end{itemize}
	\end{itemize}
\end{frame}

\begin{frame}
	\frametitle{A short history of automated story generation}
	\begin{itemize}
		\item Propp's `Morphology of the Folktale'
		\item J.R. Meehan's TALE-SPIN (story grammar)
		\item McIntyre and Lapata (learn schemata for a protagonist)
		\item Liu and Singh's MAKEBELIEVE (use common sense databases to tell stories)
		\item Riedl and Stern's IN-TALE (characters as agents)
	\end{itemize}
\end{frame}

\begin{frame}
	\frametitle{Research question}
	\begin{itemize}
		\item Research question: ``Is it possible to design a
		representation of stories and story rules in such a way that it is both
		well-suited to extracting story rules from stories and well-suited
		to generate stories from these rules?''
	\end{itemize}
\end{frame}
