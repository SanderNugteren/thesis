\section{Story and structure}

To properly represent stories a theoretical model of narratives is needed first.
The narrative model adapted for use as a basis for this thesis can be seen in figure \ref{fig:chatman}, and in this
section the different components of the model will be explained.
According to Chatman (\cite{chatman1980story}), a narrative has the following
parts: the
\emph{story (content)} and the \emph{discourse (expression)}. The story describes the events
happening in the story and the world that the story takes place in, while the
discourse covers how the content is communicated.
This thesis mainly aims to model the story, though of course the story rules that are
extracted are obtained from the story synopses in text form, so the medium, in
our case text, does have some influence on the extraction process (and since the
medium is part of the discourse, the extraction process is influenced by the
discourse).

\emph{Story events} have characters that are involved with them. These events
can either be \emph{happenings} that occur without the characters intentionally trying to
make them happen (for example, the start of a storm or an earthquake), or
\emph{actions}
that happen because the characters make them happen (for example, a character
tries to obtain something, or a character tries to persuade someone to do
something).

Another part of the story is the set of \emph{existents}, which covers
everything that exist in
the story. This includes the \emph{characters} of the story and the
\emph{setting}. In Chatman's model, the setting includes everything besides the
characters, but it
would be useful to distinguish between the \emph{setting rules} (the laws of
physics, technology, magic, law, etc. in this particular setting, but also genre
conventions that are usually present more implicitly in the story) and the \emph{setting objects} (all
objects in the story that are not characters). This would be useful since it is possible to write
many stories with different characters and setting objects in the same setting,
which implies using the same setting rules. In a broader context, it would allow
comparison between the setting rules from different stories, allowing them to be
clustered according to how similar their setings are, which could be used to
predict if stories are in similar genres.

In the original model the \emph{substance of content} was also defined, meaning
the story content filtered through the authors cultural codes. This
of course influences the existents and events, but since it is something that is
not explicit in the story it is difficult to extract in an objective manner
(compared to for example the story events). It is, however, implicitly present in
the setting rules, and so influences all narrative elements indirectly. Because
of its subjectiveness in an otherwise structuralist approach, it is therefore
not modelled in the representation presented in this thesis.

\begin{figure}
\Tree
[.Narrative
	[.{Story (content)}
		[.Events
			Actions
			Happenings
		]
		[.Existents
			Characters
			[.Setting
				{Setting rules}
				{Setting objects}
			]
		]
	]
	{Discourse (expression)}
]
\caption{Adapted from Chatman \cite{chatman1980story}}
\label{fig:chatman}
\end{figure}
