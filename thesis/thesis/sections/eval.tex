\section{Evaluation}

The system has been tested on two stories: `The Frog Prince' and
`Caliph Stork'. The story synopses of these stories were taken from Wikipedia
and annotated so they conformed to the state-action model proposed by this thesis.
These stories were chosen since they are both from the same genre (fairy tales)
and both stories contain a common story element (human to animal transformations). In `The
Frog Prince' the titular prince starts as a frog and eventually becomes a human
prince, in `Caliph Stork' the caliph and his vizier start as humans, are changed
into storks with a magic powder and both have to find a way to get back to human
form. Since the stories have this common element, the program has to be able to
handle this appropriately (extract similar rules for each occurence).

One of the problems was that the transformations were slightly different. In
Caliph Stork, the transformation was done with a magic powder, and if the user
laughed while in animal form, they could not change back. This is explicitly
told to the caliph, and by extension, to the reader. However, in `The Frog
Prince'
there are also certain prerequisites for the frog to become a prince again, but
these are never explicitly stated (did the frog have to be kissed by a princess,
or would any person, male or female, do?).

To evaluate if the program was able to capture this information, it was given
both stories with the states annotated in graph form and the actions in
simplified natural language. The only other information it was given was the
grammar to parse the action sentences.

Both stories contain 8 occurences of a transformation. There is one animal to human
transformation in `The Frog Prince', and there are 4 transformations from human 
to animal and 3 transformations from animal to human in `Caliph Stork'.
The example from `The Frog Prince' looks as follows:

\begin{verbatim}
OBJ  = `human'
SUB  = `frog'
PREC = [
    (`SUB', `SUB_race:animal'),
    (`SUB_abilities:ball', `SUB'), 
    (`SUB', `SUB_abilities:ball'),
    (`SUB_race:animal', `SUB')
    ]
EFF = {
    `edges_gone': [
        (`SUB', `SUB_race:animal')
        ],
    `edges_appeared': [
        (`SUB_race:OBJ', `SUB')
        ]
    }
\end{verbatim}

A similar situation, where the caliph in `Caliph Stork' transforms from animal 
to human form is given here:

\begin{verbatim}
OBJ  = `human'
SUB  = `caliph'
PREC = [
    (`SUB_abilities:magic_powder', `SUB'),
    ('enemy:1', `kaschnur'),
    (`SUB_abilities:magic_powder',
    `rule:know(x,magic_word)&&speak(x,magic_word)&&animal(x)->OBJ(x)'),
    (`SUB_abilities:magic_powder',
    `rule:know(x,magic_word)&&speak(x,magic_word)&&OBJ(x)->animal(x)'),
    (`SUB_abilities:magic_powder', `rule:animal(x)&&laugh(x)->not(know(x, magic_word))'),
    (`SUB_abilities:wealth', `SUB'), 
    (`SUB', 'enemy:1'), 
    (`SUB', `SUB_race:animal'),
    (`SUB', `SUB_abilities:wealth'),
    (`SUB', `SUB_abilities:know(SUB, magic_word)'),
    ('enemy:1', `SUB'), 
    (`SUB_abilities:know(SUB, magic_word)', `SUB'),
    (`SUB', `SUB_abilities:magic_powder'), 
    (`SUB_race:animal', `SUB')
    ]
EFF = {
    `edges_gone': [
        (`SUB', `SUB_race:animal')
        ],
    `edges_appeared': [
        (`SUB', 'enemy:1'),
        (`SUB_abilities:wealth', `SUB'),
        (`kaschnur', 'enemy:1'),
        (`SUB_race:OBJ', `SUB')
        ]
    }
\end{verbatim}

Another example of the same rule is from earlier in the story, when the caliph
transforms from human to animal by using a magical powder:

\begin{verbatim}
OBJ    = `animal'
SUB    = `caliph'
PREC   = [
    (`SUB_abilities:know(SUB, magic_word)', `SUB'),
    (`SUB_abilities:magic_powder', `SUB'),
    (`SUB', `SUB_race:human'),
    (`SUB_abilities:magic_powder', 
    `rule:know(x,magic_word)&&speak(x,magic_word)&&OBJ(x)->human(x)'), 
    (`SUB_race:human', `SUB'),
    (`SUB_abilities:lord_of_baghdad', `SUB'),
    (`SUB_needs:speak_with_animals', `SUB'),
    (`SUB_abilities:wealth', `SUB'),
    (`SUB', `SUB_abilities:wealth'),
    (`SUB', `SUB_abilities:know(SUB, magic_word)'),
    (`SUB_abilities:magic_powder',
    `rule:know(x,magic_word)&&speak(x,magic_word)&&human(x)->OBJ(x)'),
    (`SUB', `SUB_needs:speak_with_animals'),
    (`SUB', `SUB_abilities:magic_powder'),
    (`SUB', `SUB_abilities:lord_of_baghdad'),
    (`SUB_abilities:magic_powder', 
    `rule:OBJ(x)&&laugh(x)->not(know(x, magic_word))')
    ]
effects: {
    `edges_gone': [
        (`SUB', `SUB_race:human')
        ],
    `edges_appeared': [
        (`SUB_abilities:wealth', `SUB'),
        (`SUB_race:OBJ', `SUB')
        ]
    }
\end{verbatim}

There are similarities between all three occurences. They all share one of the
effects, namely that the race of the subject becomes the object
(\texttt{`SUB\_race:OBJ', `SUB')}.
This is a nice property, since this rule can be applied generally for all
transformations in future stories, even if the object that the person has been
transformed in has not been observed in another story.

Another common property of these rules is that the subject loses their own race
(this is denoted by \texttt{(`SUB', `SUB\_race:animal')} in the first two examples and 
\texttt{(`SUB', `SUB\_race:human')} in the third example). However, this similarity is
less pronounced than the previous one, since the race the character started out
with was not specified in the action. The first example was obtained from the
story event \texttt{`the frog transforms into a human.'}. If the story event had
consistently been in the form of \texttt{`the
frog transforms from an animal into a human.'}, then the resulting rule that
would have been learned would have been more general like the previous rule, 
since human to animal
and animal to human transformations would have a similar rule due to the
variable substitution.

The transformations in `Caliph Stork' were different from the one in `The Frog
Prince', since in `Caliph Stork' there was a magical powder with its associated
rules (the user had to speak a magical word to transform, and would forget the
word if they laughed in animal form), while in `The Frog Prince' there was no
explicit rule governing the magic. The rules of this magic powder can be seen in
the preconditions, since they were also represented as rule nodes connected to
the ones who used the powder to tranform. For `The Frog Prince' there was no
specific rule, so there are no additional rules added to the precondition. In a
generation scenario, the program could choose between these examples, either
generating a scenario with complex magical rules (as in `Caliph Stork') or a 
scenario without these rules (like in `The Frog Prince'). 

This highlights that
though the knowledge base is able to recognize regularities in the data (like
the changing of the subject's race), it is also able to represent diversity in
its rule system (a generation program has access to diverse examples of
variations within the same rule).

Of course, the evaluation has been done on just two stories, since the
annotation for now is a time-intensive process. Because of this no actual story
generation has been done, but this thesis should be seen as more of a proof of
concept: patterns can be recognized, and if more annotated stories were
available (either through annotation or, more feasibly, through more
sophisticated natural language processing), the program could learn more interesting patterns. When it is
possible to automatically annotate the stories, it should be more feasible to
generate actually interesting stories.
