\section{Conclusion and future work}

In this paper it is shown that the model can capture the preconditions and
implications of actions. Also, since the model keeps a list of probable actors
around, it can capture how these characters would act in a story. Because of the
way the model is queried however, these are not rules set in stone. A princess
might not murder often, but there is a small chance that it does happen. In
fact, a story where the princess would murder could be considered original.

Creativity should be different enough from the historical data to be be novel,
but not so different that the story makes no sense. Since the model can be
easily queried for probabilities of certain story actions involving certain
people, a planning algorithm could try to stay within these boundaries, by
using an unconventional action when the story is becoming too predictable, or a
conventional one when the story is becoming too bizarre.

The current approach uses variables to keep track of how many times a certain
character was involved with an action and in what way. In a fairy tale context
this works, since characters are usually not named but just know by their
archetype (the princess, the king, the witch, etc.). One could argue that this
is just a peculiarity of fairy tales, but other genres have certain conventions
too. For example, in film noir, there are also certain stock characters (the
hard-boiled detective, the mysterious nightclub singer), but usually these are
known by their names, not by their archetype. An extension of this algorithm
could have characters with attributes, and let the probabilities be dependent on
the character attributes instead of the characters themselves.

Another way this approach could be extended is to get some more sophisticated
natural language parsing being done. The stories the current model was trained
on were manually annotated, but using only the information available from the
story synopses available from Wikipedia or similar sources. The challenges here
would be parsing the natural language, recognize the entities and how they
interact. This would still be easier than feeding the algorithm the complete
book, since then the algorithm would also have to filter out which parts are
actually rellevant for the plot.
