\section{Related Work}

Much research has been done on computational creativity, for example in
videogames (\cite{liapis2014computational} \cite{crawford1998computer}), painting
(\cite{colton2015painting}), dance choreography (\cite{carlson2011scuddle}) and
music (\cite{johnson2014musical}).

Most early work in storytelling has been inspired by the work of Propp
(\cite{propp1968morphology}). Propp identified many recurring plot elements in
Russian folk tales (for example `pursuit', where the hero pursues the villain, or
`wedding', where the hero marries and gets rewarded by the community (for example
by ascension to the throne)), as well as recurring characters (hero, villain,
the helper and the princess). This served to create an ontology for folk tales,
so that they could be more easily analyzed and compared, but was later used as a
framework by approaches using a story grammar.

More recently, there
is the work by Gerv\'as et al. (\cite{Gervas2005235}) that uses case-based
reasoning with a representation based on the plot elements and character
archetypes of Propp. The way this works is that a sequence of events is generated using the
building blocks provided by Propp, and then templates are used to generate a
story in natural language.
The model proposed in this thesis is more general, however, since there is very
little domain knowledge involved (just the content of the stories themselves, as
opposed to all the knowledge from Propp).
This makes this model more generally applicable outside of the folk tale domain,
not just for generating new stories within one domain, but possibly also to
build a story model using multiple genres at the same time.

Ther is also the work of McIntyre and Lapata (\cite{McIntyre2010}), which
generates stories using evolutionary algorithms. This work is inspired by the
work of Chambers and Jurafsky (\cite{chambers2009unsupervised}). Both works are
also datadriven, but they try to learn certain predicate co-occurences (when someone
gets arrested, he is usually interrogated next) and entity-predicate
co-occurences (the person doing the arresting is a \texttt{\{detective, police
officer\}}). The schemata learned in this way are learned for events involving a
certain participant (called the \emph{protagonist}). This leads to schemata that
can contain a great deal of events. This schemata can be branching (somebody on trial
might be convicted or go free), further increasing the complexity of the
structures learned.
Though generating these long event chains would improve the coherence of a story, 
it might impede the creativity of the story generator, since trying something novel is
difficult to do with these rigid structures, while the structures presented in
this work are much simpler and therefore more easily recombined in novel ways.
%I dunno, maybe this is true

%TODO add something about agent-based storytelling?

The work in this paper is mainly inspired by the MEXICA story generator. %TODO cite
However, in MEXICA the relations between characters were much simpler, and 
all possible actions had to be pre-defined. In this work, the focus is more on
the rule-extraction side, whereas MEXICA was more focused on the generation of
the stories. The method described in this paper could be used to suplement
MEXICA with a way to learn actions from stories themselves, while leaving the
actual generation to MEXICA itself, though.
