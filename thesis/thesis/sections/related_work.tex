\section{Related Work}

Much research has been done on computational creativity, for example in
videogames (\cite{liapis2014computational} \cite{crawford1998computer}), painting
(\cite{colton2015painting}), dance choreography (\cite{carlson2011scuddle}) and
music (\cite{johnson2014musical}).

In this section related work on storytelling will be discussed. First the 
different approaches
to story analysis and structure will be covered, followed by a discussion of story
representation. Afterwards some different formalisms for story generation will be
discussed.

\subsection{Story}

There are many movements of literary theory.
For example, there is the Marxist theory of literary criticism, which views
stories in terms of the society from which they originate
(\cite{eagleton2002marxism}), Darwinian literary criticism which views stories
in the context of evolutionary biology (\cite{carroll2004literary}) and many others.
One such movement is the Russian
formalist movement, which was a structuralist movement. The structuralists
tried to analyze literature by analyzing its structure (the story
devices that were used) instead of the previously mentioned traditional 
psychological or
culture-historical approaches. Because of the emphasis on story structure this
school of literary theory is the most attractive starting point for automated story telling, since
the structures are already catalogued in such a way that a grammar or similar
rule system can be based on them. This framework formed the basis for many early
automated story generators, which will be discussed in section 2.3.
The work of Chatman (\cite{chatman1980story})
can be seen in a similar structuralist vein, since it describes stories in terms
of their structure too. Chatman defines a hierarchy of narrative elements such
as characters, actions and setting, which together form the whole narrative
experience. Chatman's hierarchy will be explored further in section 3.

\subsection{Story representation}

In 1975, Minsky devised a representation of events called \emph{frames}
(\cite{minsky1975framework}). These frames represented a stereotyped scenario
(such as going to a child's birthday party), with certain concepts being constants
(always true for that type of situation) and others being variables that could
be instantiated for each situation. Related frames (representing sequences of
scenarios) are linked together in \emph{frame systems}. This representation was
not devised specifically for storytelling purposes, but rather to capture
general world knowledge, from natural language to computer vision, in such a way
that these diverse types of knowledge could be compared. Because of this the
representation is described in very general terms, and therefore difficult to
apply in practice.

A similar but more practical version of this idea was proposed by Schank and Abelson
\cite{schank1975scripts}. They
tried to capture similar sequences of events in \emph{scripts}. This
representation was similar to the frame representation, but was more structured,
since all possible actions were defined. Since the aim was to represent
knowledge and not storytelling, it did not implement measures to ensure
creativity, but was geared more towards question answering about the story
content.

A continuation of this work on knowledge representation can be seen in case-based reasoning.
Case-based reasoning is based on the idea that new problems can be solved by
using the solutions to past problems. This is done by adapting the solution of
an old problem to fit a new problem. This idea was pioneered by Schank
(\cite{schank1983dynamic}).

To aid the creation of the semantic web (a machine-readable internet), another
form of knowledge representation was developed, the Resource Description
Framework (RDF) (\cite{klynecarrollrdf}). This representation specification was
designed to represent the relations between different entities (which were
represented by uniform resource identifier (URI)). These relations were
represented as so-called triples, in the form of subject-predicate-object. The
ontologies created with RDF can be seen as graphs, with URI's being the nodes
and predicates being the relations between these nodes.

\subsection{Story generation}

Most early work in automated storytelling has been inspired by the work of Propp
(\cite{propp1968morphology}), who was a Russian formalist. Propp identified many recurring plot elements in
Russian folk tales (for example `pursuit', where the hero pursues the villain, or
`wedding', where the hero marries and gets rewarded by the community (for example
by ascension to the throne)), as well as recurring character archetypes (hero, villain,
the helper and the princess).

The work of Propp served to create an ontology for folk tales,
so that they could be more easily analyzed and compared, but later inspired
approaches using a story grammar, the intuition being that there exists a distinct
ordering of story elements for all stories (with variations being possible in
story elements and some story elements being optional). A
notable example of this story grammar philosophy is TALE-SPIN
(\cite{meehan1977tale}), which tries to model the author process as generating
a problem for the story by selecting certain attributes for characters (for example, one of the characters is hungry), and then
applies certain grammar rules to solve the problem of the story (the character
tries to obtain food).
Of course, all story rules and all possible character attributes for the grammar were put in
a priori, and therefore the stories that could be generated were very basic. Also,
story generation is seen as purely logic problem solving, and therefore there
is no way to distinguish between all possible problem solutions in such a way that the
most enjoyable solution (for the reader) is chosen.

More recently, there
is the work by Gerv\'as et al. (\cite{Gervas2005235}) that uses case-based
reasoning with a representation based on the plot elements and character
archetypes of Propp.
The way the approach by Gerv\'as works is that a sequence of events is generated using the
building blocks provided by Propp, and then templates are used to generate a
story in natural language.
%TODO move this to a later section
The model proposed in this thesis is more general, however, since there is very
little domain knowledge involved (just the content of the stories themselves, as
opposed to the manually obtained knowledge from Propp).
This makes this model more generally applicable outside of the folk tale domain,
not just for generating new stories within other genres by simply training on a
different set of stories, but possibly also to
build a story model using multiple genres at the same time.

McIntyre and Lapata (\cite{McIntyre2010}) describe a system which
generates stories using evolutionary algorithms. This work is inspired by the
work of Chambers and Jurafsky (\cite{chambers2009unsupervised}). Both works are
also datadriven, but they try to learn certain predicate co-occurences (when someone
gets arrested, he is usually interrogated next) and entity-predicate
co-occurences (the person doing the arresting is a \texttt{\{detective, police
officer\}}). The schemata learned in this way are learned for events involving a
certain participant (called the \emph{protagonist}). This leads to schemata that
can contain a great deal of events. This schemata can be branching (somebody on trial
might be convicted or go free), further increasing the complexity of the
structures learned.
Though generating stories using these long event chains would improve story coherence,
it might impede the creativity of the story generator, since trying something novel is
difficult to do with these rigid structures (not only are the structures very
large, making them difficult to recombine, but they are also centered on the
protagonist, so substituting the protagonist becomes difficult), while the structures presented in
this work are much simpler and therefore more easily recombined in novel ways.

The storygenerator MAKEBELIEVE by Liu and Singh (\cite{liu2002makebelieve}) uses
a database of common sense reasoning to generate its stories. It would start out
with some initial premise, and then use common sense to infer the logical
consequences. However, this common sense
database was not specifically constructed to generate stories, so the stories
that were generated tended to be very logical, but therefore also very
predictable. By
learning from the stories directly, we know that the events have appeared at
least once in a story, and therefore can be used to make a compelling story.
In addition to this, these common sense databases were generally created to
solve real-world problems. However, in many story genres
the rules of what is possible in the world are very different. In fairy tales,
there is magic which can do things that are not normally possible, while in science fiction
advanced technology fulfills a similar function. Because of this, these common sense
databases are ill-suited to generate these kinds of stories, which brings back
the problem that somehow knowledge again has to be obtained about the fantasy
world (and if the representation of that knowledge is still in ontologies,
these ontologies would have to be created for each genre or even each individual
story (the rules from a science fiction can be different from fairy tales, and
even between two separate science fiction universes there might be a difference
in what is and what is not possible)).

%TODO add something about agent-based storytelling?
Another paradigm that has gained popularity in the field of automated story
generation is an agent-based approach. In this approach, all characters are
logical agents interacting with eachother. In some cases there is also a
director-like agent, that ensures that the story that is being generated adheres
to certain constraints to make the story enjoyable, instead of just a long
series of interactions between agents.

In a videogame or a training simulation, where the user interacts with different
entities all the time, this is a very natural way of thinking about story
generation. For example, IN-TALE by Riedl and Stern (\cite{riedl2006believable})
was used to simulate the experience of being an army officer of a peacekeeping force
on a busy marketplace in a 3D computer game engine. In this simulation, each
character is an agent with its own goals, trying to achieve them within the
simulation while interacting with the human character. Another example is the
work of Cavazza, Charles and Mead (\cite{cavazza2002interacting}), where the
user is similarly able to influence the narrative by her own actions. However,
both these systems have in common that they work off a certain scenario,
constricting the possible actions of both the user and the actors. Because of
this, these systems are not able to generate a large variety of different
stories, but just variations within the same scenario.
%It would be a stretch to call these systems creative.

The work in this paper is mainly inspired by the MEXICA story generator
(\cite{perez2001mexica}).
The architecture used in this thesis is similar, in that it also represents the
story as a sequence of states and actions, and that each state consists of a
graph of all actors in the story. This state representation is similar to the
RDF format, since all relations are also defined as triples
(subject-predicate-object).
However, in MEXICA the relations between characters were much simpler (the only
type of relation between characters is a 5 point scale ranging from hatred to
love), and 
all possible actions had to be pre-defined (they were not learned from other
stories, like in this thesis). In this work, the focus is more on
the rule-extraction side, whereas MEXICA was more focused on the generation of
the stories. The method described in this paper could be used to suplement
MEXICA with a way to learn actions from stories themselves, while leaving the
actual generation to MEXICA itself, though.
