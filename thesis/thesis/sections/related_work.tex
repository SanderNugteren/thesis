\section{Related Work}

There have been many approaches to computational creativity. For example, there
is the work by Gerv\'as et al. (\cite{Gervas2005235}) that uses case-based
reasoning with a representation based on the work of Propp (\cite{propp1968morphology}).
The model proposed in this paper is more general, however, since there is very
little domain knowledge involved (just the content of the stories themselves).
This makes this model more generally applicable outside of the folk tale domain.

The work in this paper is mainly inspired by the MEXICA story generator. %TODO cite
However, in MEXICA the relations between characters were much simpler, and 
all possible actions had to be pre-defined. In this work, the focus is more on
the rule-extraction side, whereas MEXICA was more focused on the generation of
the stories.
