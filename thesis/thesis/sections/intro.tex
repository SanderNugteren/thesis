\section{Introduction}
%Creativity is important

Creativity is seen as being closely correlated with intelligence (either
creativity being a component of intelligence, or the other way around).
However, many feel that computer programs cannot be creative. Because of the
previously outlined correlation, this also discounts the field of artificial
intelligence by stating it can never be ``true'' intelligence.

According to Boden (\cite{Boden1998347})
there are two types of creativity. There is psychological creativity (or
\emph{p-creativity}), where an idea is considered novel on a personal level, and
there is historical creativity (\emph{h-creativity}), where the idea has to be
novel on a historical level (i.e., have other people in history come up with the
same idea?).
In computational creativity, usually a form of p-creativity is considered, since
to have an h-creative system would require to put in all history of that
subject. However, this does not mean that algorithms can not produce h-creative
work.

There are many ways in which creativity can be expressed. In fact, any idea that
is considered to be novel and to have some kind of value (e.g., monetary, aesthetic) is
considered an expression of creativity. However, the term creativity is mostly used with regards to
the arts, such as painting, sculpting, poetry and storytelling. In these
artistic fields, creativity can be seen as a problem with two opposing
objectives. On the one hand, the idea has to be novel, which means that it must be different from
the existing works. On the other hand, it should still be comprehensible with
respect to the rules and conventions of the art. Generating a story filled with
gibberish or a painting with random pixels is not a valuable contribution to the
art, since it is hard to interpret a meaning. Of course, this is an extreme
case, but one could also say that if a writer set out to create a new work in a
particular genre (for example, a fairy tale), but ended up with something that
is not recognized as a work belonging to that genre, he failed at his task to
create a work in that genre (even though the work might have some unintentional
value in some other way).

%Storytelling is important
Storytelling is an important part of human cultures around the world.
Stories are used to preserve history and culture, and are used to transfer values
and experiences across generations, encouraging a broadening of horizons through
identifying with the characters in the story, and relating these experiences to
the real world around them. The universalness of storytelling seems to imply
that it is something fundamentally human to do, which might tie it (like
creativity) into the human thought process.
%TODO cut down on/rewrite bullshit

The approach used in this thesis is a form of instance-based learning, where the
rules are learned from the stories in the story database themselves. This has the advantage that
the measure of creativity of a rule is dependent on which stories have already been seen,
which is a good metric if one considers p-creativity. Furthermore, the more data
will be added, the more the rule distribution of the model will start to match 
the actual historical use of rules and tropes, which would bring the system closer to
h-creativity.

From another perspective, one could say that this is analogous to how human
creative writing process works. Authors read a lot of stories or have real-life
experiences and recombine elements in their own stories in novel and exciting ways.

The ruleset extracted in this way can be used both in a fully automated story
generator, where an algorithm uses the rules and their associated probabilities
to generate a story with minimal human input or even none at all, or it could be used as a component
of a recommender system for writers (like the co-creative approaches detailed in
\cite{kantosalo2014isolation}), where it recieves the story the author
wrote and makes suggestions to help further plot development, or augment the
story to make it more interesting. It could also be used to compare different
stories and analyze where similarities can be found. This would lead to
automated story analysis, which could help writers in a more indirect way (by
learning more about the structure of other stories, a writer can improve their
understanding of a particular genre or storytelling in general).
