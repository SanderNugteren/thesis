\section{Introduction}
%Storytelling is important
%Creativity is important
According to ... %TODO cite source
there are two types of creativity. There is personal creativity (or
\emp{p-creativity}), where an idea is considered novel on a personal level, and
there is historical creativity (\emp{h-creativity}), where the idea has to be
novel on a historical level (i.e., have other people in history come up with the
same idea?).
In computational creativity, usually a form of p-creativity is considered, since
to have an h-creative system would require to put in all history of that
subject.

The approach used in this paper is a form of instance-based learning, where the
rules are learned from the story database itself. This has the advantage that
the creativity of a rule is dependent on which stories have already been seen,
which is a good metric if one considers p-creativity. Furthermore, the more data
will be added, the more the rule distribution of the model will start to match 
the actual historical use of the rules, which would bring the system closer to
h-creativity.
