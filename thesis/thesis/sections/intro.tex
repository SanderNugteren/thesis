\section{Introduction}
%Storytelling is important
%Creativity is important
Creativity is seen as being closely correlated with intelligence (either
creativity being a component of intelligence, or the other way around).
However, many feel that computer programs cannot be creative. Because of the
previously outlined correlation, this also discounts the field of artificial
intelligence by stating it is not "truly" intelligent.

According to Boden (\cite{Boden1998347})
there are two types of creativity. There is psychological creativity (or
\emph{p-creativity}), where an idea is considered novel on a personal level, and
there is historical creativity (\emph{h-creativity}), where the idea has to be
novel on a historical level (i.e., have other people in history come up with the
same idea?).
In computational creativity, usually a form of p-creativity is considered, since
to have an h-creative system would require to put in all history of that
subject. However, this does not mean that algorithms can not produce h-creative
work.

%Many forms of creativity (painting, sculpting, story generation)

The approach used in this paper is a form of instance-based learning, where the
rules are learned from the story database itself. This has the advantage that
the creativity of a rule is dependent on which stories have already been seen,
which is a good metric if one considers p-creativity. Furthermore, the more data
will be added, the more the rule distribution of the model will start to match 
the actual historical use of the rules, which would bring the system closer to
h-creativity.

From another perspective, one could say that this is analogous to how human
writers write (they read a lot of stories or have real-life experiences and 
recombine elements in novel and exciting ways).
