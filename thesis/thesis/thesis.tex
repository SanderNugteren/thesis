\documentclass{article}
\usepackage{graphicx, color}
\usepackage[a4paper,margin=2cm]{geometry}
\usepackage{tikz}
\usetikzlibrary{arrows.meta,positioning,calc}

\newcommand{\red}[1]{{\color{red}{#1}}}
\let\oldsection\section
\renewcommand\section{\clearpage\oldsection}

\begin{document}


\begin{titlepage}

\newcommand{\HRule}{\rule{\linewidth}{0.5mm}} % Defines a new command for the horizontal lines, change thickness here
\center % Center everything on the page
 
%----------------------------------------------------------------------------------------
%	HEADING SECTIONS
%----------------------------------------------------------------------------------------

\includegraphics[width=\linewidth]{../images/uvaENG}\\[2.5cm]
\textsc{\Large MSc Artificial Intelligence}\\[0.2cm]
\textsc{\normalsize Track: Learning Systems}\\[1.0cm] % track
\textsc{\Large Master Thesis}\\[0.5cm] 

%----------------------------------------------------------------------------------------
%	TITLE SECTION
%----------------------------------------------------------------------------------------

\HRule \\[0.4cm]
{ \huge \bfseries Extracting rules from story synopses}\\[0.4cm] % Title of your document
\HRule \\[0.5cm]
 
%----------------------------------------------------------------------------------------
%	AUTHOR SECTION
%----------------------------------------------------------------------------------------

by\\[0.2cm]
\textsc{\Large Sander Nugteren}\\[0.2cm] %you name
6042023\\[1cm]


%----------------------------------------------------------------------------------------
%	DATE SECTION
%----------------------------------------------------------------------------------------

{\Large \today}\\[1cm] % Date, change the \today to a set date if you want to be precise

42 EC\\ %
June 2016\\[1cm]%

%----------------------------------------------------------------------------------------
%	COMMITTEE SECTION
%----------------------------------------------------------------------------------------
\begin{minipage}[t]{0.4\textwidth}
\begin{flushleft} \large
\emph{Supervisor:} \\
Dr. F.M. \textsc{Nack}% Supervisor's Name
\end{flushleft}
\end{minipage}
~
\begin{minipage}[t]{0.4\textwidth}
\begin{flushright} \large
\emph{Assessor:} \\
Dr. R.G.F. \textsc{Winkels}\\
\end{flushright}
\end{minipage}\\[2cm]

%----------------------------------------------------------------------------------------
%	LOGO SECTION
%----------------------------------------------------------------------------------------

%\framebox{\rule{0pt}{2.5cm}\rule{2.5cm}{0pt}}\\[0.5cm]
%\includegraphics[width=2.5cm]{figure}\\ % Include a department/university logo - this will require the graphicx package
%\textsc{\large \red{institute name}}\\[1.0cm] % 
 
%----------------------------------------------------------------------------------------

\vfill % Fill the rest of the page with whitespace

\end{titlepage}

\tableofcontents

\clearpage

\begin{abstract}
This work aims to develop a representation of stories, to facilitate learning the structure of stories, in
such a way that these existing stories can be used to generate new stories. When trained on two fairy tales
(“The Frog Prince” and “Caliph Stork”) the program was able to capture the rules of the stories and was
able to generalize over multiple occurences of the same event type in different stories (in this case, magical
transformation from animal to human and vice versa).
\end{abstract}

\section{Introduction}
%Creativity is important

Creativity is seen as being closely correlated with intelligence (either
creativity being a component of intelligence, or the other way around).
However, many feel that computer programs cannot be creative. Because of the
previously outlined correlation, this also discounts the field of artificial
intelligence by stating it can never be ``true'' intelligence.

According to Boden (\cite{Boden1998347})
there are two types of creativity. There is psychological creativity (or
\emph{p-creativity}), where an idea is considered novel on a personal level, and
there is historical creativity (\emph{h-creativity}), where the idea has to be
novel on a historical level (i.e., have other people in history come up with the
same idea?).
In computational creativity, usually a form of p-creativity is considered, since
to have an h-creative system would require to put in all history of that
subject. However, this does not mean that algorithms can not produce h-creative
work.

There are many ways in which creativity can be expressed. In fact, any idea that
is considered to be novel and to have some kind of value (e.g., monetary, aesthetic) is
considered an expression of creativity. However, the term creativity is mostly used with regards to
the arts, such as painting, sculpting, poetry and storytelling. In these
artistic fields, creativity can be seen as a problem with two opposing
objectives. On the one hand, the idea has to be novel, which means that it must be different from
the existing works. On the other hand, it should still be comprehensible with
respect to the rules and conventions of the art. Generating a story filled with
gibberish or a painting with random pixels is not a valuable contribution to the
art, since it is hard to interpret a meaning. Of course, this is an extreme
case, but one could also say that if a writer set out to create a new work in a
particular genre (for example, a fairy tale), but ended up with something that
is not recognized as a work belonging to that genre, he failed at his task to
create a work in that genre (even though the work might have some unintentional
value in some other way).

%Storytelling is important
Storytelling is an important part of human cultures around the world.
Stories are used to preserve history and culture, and are used to transfer values
and experiences across generations, encouraging a broadening of horizons through
identifying with the characters in the story, and relating these experiences to
the real world around them. The universalness of storytelling seems to imply
that it is something fundamentally human to do, which might tie it (like
creativity) into the human thought process.
%TODO cut down on/rewrite bullshit

The approach used in this thesis is a form of instance-based learning, where the
rules are learned from the stories in the story database themselves. This has the advantage that
the measure of creativity of a rule is dependent on which stories have already been seen,
which is a good metric if one considers p-creativity. Furthermore, the more data
will be added, the more the rule distribution of the model will start to match 
the actual historical use of rules and tropes, which would bring the system closer to
h-creativity.

From another perspective, one could say that this is analogous to how human
creative writing process works. Authors read a lot of stories or have real-life
experiences and recombine elements in their own stories in novel and exciting ways.

The ruleset extracted in this way can be used both in a fully automated story
generator, where an algorithm uses the rules and their associated probabilities
to generate a story with minimal human input or even none at all, or it could be used as a component
of a recommender system for writers (like the co-creative approaches detailed in
\cite{kantosalo2014isolation}), where it recieves the story the author
wrote and makes suggestions to help further plot development, or augment the
story to make it more interesting. It could also be used to compare different
stories and analyze where similarities can be found. This would lead to
automated story analysis, which could help writers in a more indirect way (by
learning more about the structure of other stories, a writer can improve their
understanding of a particular genre or storytelling in general).

\section{Related Work}

Much research has been done on computational creativity, for example in
videogames (\cite{liapis2014computational} \cite{crawford1998computer}), painting
(\cite{colton2015painting}), dance choreography (\cite{carlson2011scuddle}) and
music (\cite{johnson2014musical}).

Most early work in storytelling has been inspired by the work of Propp
(\cite{propp1968morphology}). Propp identified many recurring plot elements in
Russian folk tales (for example `pursuit', where the hero pursues the villain, or
`wedding', where the hero marries and gets rewarded by the community (for example
by ascension to the throne)), as well as recurring character archetypes (hero, villain,
the helper and the princess). This served to create an ontology for folk tales,
so that they could be more easily analyzed and compared, but later inspired
approaches using a story grammar, the intuition being that there exists a distinct
ordering of story elements for all stories (with some story elements being optional). A
notable example of this story grammar philosophy is TALE-SPIN
(\cite{meehan1977tale}), which tries to model the author process as generating
a problem for the story by selecting certain attributes for characters (for example, one of the characters is hungry), and then
applies certain grammar rules to solve the problem of the story (the character
tries to obtain food).
Of course, all story rules and all possible character attributes for the grammar were put in
a priori, and therefore the stories that were generated were very basic. Also,
story generation is seen as purely logic problem solving, and therefore there
is no way to distinguish between all possible problem solutions in such a way that the
most enjoyable solution (for the reader) is chosen.

More recently, there
is the work by Gerv\'as et al. (\cite{Gervas2005235}) that uses case-based
reasoning with a representation based on the plot elements and character
archetypes of Propp. The way this works is that a sequence of events is generated using the
building blocks provided by Propp, and then templates are used to generate a
story in natural language.
The model proposed in this thesis is more general, however, since there is very
little domain knowledge involved (just the content of the stories themselves, as
opposed to the manually obtained knowledge from Propp).
This makes this model more generally applicable outside of the folk tale domain,
not just for generating new stories within other genres by simply training on a
different set of stories, but possibly also to
build a story model using multiple genres at the same time.

There is also the work of McIntyre and Lapata (\cite{McIntyre2010}), which
generates stories using evolutionary algorithms. This work is inspired by the
work of Chambers and Jurafsky (\cite{chambers2009unsupervised}). Both works are
also datadriven, but they try to learn certain predicate co-occurences (when someone
gets arrested, he is usually interrogated next) and entity-predicate
co-occurences (the person doing the arresting is a \texttt{\{detective, police
officer\}}). The schemata learned in this way are learned for events involving a
certain participant (called the \emph{protagonist}). This leads to schemata that
can contain a great deal of events. This schemata can be branching (somebody on trial
might be convicted or go free), further increasing the complexity of the
structures learned.
Though generating these long event chains would improve the coherence of a story, 
it might impede the creativity of the story generator, since trying something novel is
difficult to do with these rigid structures (not only are the structures very
large, making them difficult to recombine, but they are also centered on the
protagonist, so substituting the protagonist becomes difficult to do right), while the structures presented in
this work are much simpler and therefore more easily recombined in novel ways.

The storygenerator MAKEBELIEVE by Liu and Singh (\cite{liu2002makebelieve}) uses
a database of common sense reasoning to generate its stories. However, this
database was not specifically constructed to generate stories, so the stories
that were generated tended to be very logical, but therefore also very
predictable. By
learning from the stories directly, we know that the events have appeared at
least once in a story, and therefore can be used to make a compelling story.
In addition to this, these common sense databases were generally created to
solve real-world problems. However, in many story genres
the rules of what is possible in the world are very different. In fairy tales,
there is magic which can do things that are not normally possible, while in science fiction
advanced technology fulfills a similar function. Because of this, these common sense
databases are ill-suited to generate these kinds of stories, which brings back
the problem that somehow knowledge has to be obtained about the fantasy
world again (and if the representation of that knowledge is still in ontologies,
these ontologies would have to be created for each genre or even each individual
story (the rules from a science fiction can be different from fairy tales, and
even between two separate science fiction universes there might be a difference
in what is and what is not possible)).

%TODO add something about agent-based storytelling?
Another paradigm that has gained popularity in the field of automated story
generation is an agent-based approach. In this approach, all characters are
logical agents interacting with eachother. In some cases there is also a
director-like agent, that ensures that the story that is being generated adheres
to certain constraints to make the story enjoyable, instead of just a long
series of interactions between agents.

In a videogame or a training simulation, where the user interacts with different
entities all the time, this is a very natural way of thinking about story
generation. For example, IN-TALE by Riedl and Stern (\cite{riedl2006believable})
was used to simulate the experience of being an army officer of a peacekeeping force
on a busy marketplace in a 3D computer game engine. In this simulation, each
character is an agent with its own goals, trying to achieve them within the
simulation while interacting with the human character. Another example is the
work of Cavazza, Charles and Mead (\cite{cavazza2002interacting}), where the
user is similarly able to influence the narrative by her own actions. However,
both these systems have in common that they work off a certain scenario,
constricting the possible actions of both the user and the actors. Because of
this, these systems are not able to generate a large variety of different
stories, but just variations within the same scenario.
%It would be a stretch to call these systems creative.

The work in this paper is mainly inspired by the MEXICA story generator
(\cite{perez2001mexica}). %TODO cite
The architecture used in this thesis is similar, in that it also represents the
story as a sequence of states and actions, and that each state consists of a
graph of all actors in the story.
However, in MEXICA the relations between characters were much simpler (the only
type of relation between characters is a 5 point scale ranging from hatred to
love), and 
all possible actions had to be pre-defined (they were not learned from other
stories, like in this thesis). In this work, the focus is more on
the rule-extraction side, whereas MEXICA was more focused on the generation of
the stories. The method described in this paper could be used to suplement
MEXICA with a way to learn actions from stories themselves, while leaving the
actual generation to MEXICA itself, though.

\section{Model}

\begin{frame}
	\frametitle{A pipeline for story generation}
	\begin{itemize}
		\item
	\end{itemize}
\end{frame}

\begin{frame}
	\frametitle{Story representation}
	Stories consist of:
	\begin{itemize}
		\item States
		\item Actions
		\item Sub-stories
	\end{itemize}
\end{frame}

\begin{frame}
	\frametitle{States}
	States consist of:
	\begin{itemize}
		\item 
	\end{itemize}
\end{frame}

\begin{frame}
	\frametitle{Actions}
	\begin{itemize}
		\item Sentences in (simplified) natural language (\texttt{``the frog
		gets the ball''})
		\item Parsed to extract \texttt{verb}, \texttt{subj}, \texttt{obj},
		\texttt{dat}
	\end{itemize}
\end{frame}

\begin{frame}
	\frametitle{Example of story representation}
\end{frame}

\section{Evaluation}

The system has been tested on two stories: `The Frog Prince' (\cite{frogprince}) and
`Caliph Stork' (\cite{stork}). The stories were manually
annotated so they conformed to the state-action model proposed by this thesis.
In section 5.1 the stories as annotated will be presented and the similarities
and differences between them will be discussed, and in section 5.2 an example
involving magical transformations will be shown

\subsection{Dataset}

`The Frog Prince' and `Caliph Stork' were chosen since they are both from the same genre (fairy tales)
and both stories contain a common story element (human to animal transformations). In `The
Frog Prince' the titular prince starts as a frog and eventually becomes a human
prince, in `Caliph Stork' the caliph and his vizier start as humans, are changed
into storks with a magic powder and both have to find a way to get back to human
form. Since the stories have this common element, the program has to be able to
handle this appropriately (extract similar rules for each occurence).

The actions of `The Frog Prince' were as follows:
\begin{verbatim}
the frog promises the ball to the princess
the princess promises companionship to the frog
the frog gets the ball
the princess goes home. the princess has dinner with the king
the frog asks the princess for promise:2. the princess refuses promise:2
the king enforces the promise:2
the princess uses companionship
the frog transforms into a human
the frog marries the princess
\end{verbatim}

The actions of `Caliph Stork' were the following:
\begin{verbatim}
the caliph buys the magic_powder
caliph learns about magic_powder
caliph uses magic_powder. caliph says magic_word. 
caliph uses magic_powder. caliph says magic_word. the caliph transforms into an animal
vizier uses magic_powder. vizier says magic_word. the vizier transforms into an animal
the caliph listens to an animal
the vizier listens to an animal
the caliph laughs
the vizier laughs
mirza becomes the lord_of_baghdad
the caliph meets the owl
the vizier meets the owl
SUBSTORY_BEGIN
	king_of_the_indies denies kaschnur
	the owl transforms into an animal
SUBSTORY_END
caliph learns the  magic_word
vizier learns the magic_word
caliph transforms into a human
vizier transforms into a human
caliph marries the owl
the owl transforms into a human
caliph kills kaschnur
mirza transforms into an animal
\end{verbatim}

Even though both stories contain transformations, they vary in complexity with
respect to the rules. In
Caliph Stork, the transformation was done with a magic powder, the character had to
remember a magic word, and if the character
laughed while in animal form, they forgot the magic word and could not change back. 
This is explicitly told to the caliph, and by extension, to the reader. 
However, in `The Frog Prince'
there are also certain prerequisites for the frog to become a prince again, but
these are never explicitly stated (did the frog have to be kissed by a princess,
or would any person, male or female, do?). Because of this, no explicit
annotation has been done about the rules of the magic transformation in the frog
prince, which results in a much simpler set of prerequisites compared to Caliph
Stork.

\subsection{Extraction of transformation}

To evaluate if the program was able to capture this information, it was given
both stories with the states annotated in graph form and the actions in
simplified natural language. The only other information it was given was the
grammar to parse the action sentences.

Both stories contain 8 occurences of a transformation. There is one animal to human
transformation in `The Frog Prince', and there are 4 transformations from human 
to animal and 3 transformations from animal to human in `Caliph Stork'.
The example from `The Frog Prince' (where the story event is `the frog
transforms into a human') looks as follows:

\begin{verbatim}
OBJ  = `human'
SUB  = `frog'
PREC = [
    (`SUB', `SUB_race:animal'),
    (`SUB_abilities:ball', `SUB'), 
    (`SUB', `SUB_abilities:ball'),
    (`SUB_race:animal', `SUB')
    ]
EFF = {
    `edges_gone': [
        (`SUB', `SUB_race:animal')
        ],
    `edges_appeared': [
        (`SUB_race:OBJ', `SUB')
        ]
    }
\end{verbatim}

A similar situation, where the caliph in `Caliph Stork' transforms from animal 
to human form is given here (The story event being `caliph transforms into a
human'):

\begin{verbatim}
OBJ  = `human'
SUB  = `caliph'
PREC = [
    (`SUB_abilities:magic_powder', `SUB'),
    ('enemy:1', `kaschnur'),
    (`SUB_abilities:magic_powder',
    `rule:know(x,magic_word)&&speak(x,magic_word)&&animal(x)->OBJ(x)'),
    (`SUB_abilities:magic_powder',
    `rule:know(x,magic_word)&&speak(x,magic_word)&&OBJ(x)->animal(x)'),
    (`SUB_abilities:magic_powder', `rule:animal(x)&&laugh(x)->not(know(x, magic_word))'),
    (`SUB_abilities:wealth', `SUB'), 
    (`SUB', 'enemy:1'), 
    (`SUB', `SUB_race:animal'),
    (`SUB', `SUB_abilities:wealth'),
    (`SUB', `SUB_abilities:know(SUB, magic_word)'),
    ('enemy:1', `SUB'), 
    (`SUB_abilities:know(SUB, magic_word)', `SUB'),
    (`SUB', `SUB_abilities:magic_powder'), 
    (`SUB_race:animal', `SUB')
    ]
EFF = {
    `edges_gone': [
        (`SUB', `SUB_race:animal')
        ],
    `edges_appeared': [
        (`SUB', 'enemy:1'),
        (`SUB_abilities:wealth', `SUB'),
        (`kaschnur', 'enemy:1'),
        (`SUB_race:OBJ', `SUB')
        ]
    }
\end{verbatim}

Another example of the same rule is from earlier in the story, when the caliph
transforms from human to animal by using a magical powder:

\begin{verbatim}
OBJ    = `animal'
SUB    = `caliph'
PREC   = [
    (`SUB_abilities:know(SUB, magic_word)', `SUB'),
    (`SUB_abilities:magic_powder', `SUB'),
    (`SUB', `SUB_race:human'),
    (`SUB_abilities:magic_powder', 
    `rule:know(x,magic_word)&&speak(x,magic_word)&&OBJ(x)->human(x)'), 
    (`SUB_race:human', `SUB'),
    (`SUB_abilities:lord_of_baghdad', `SUB'),
    (`SUB_needs:speak_with_animals', `SUB'),
    (`SUB_abilities:wealth', `SUB'),
    (`SUB', `SUB_abilities:wealth'),
    (`SUB', `SUB_abilities:know(SUB, magic_word)'),
    (`SUB_abilities:magic_powder',
    `rule:know(x,magic_word)&&speak(x,magic_word)&&human(x)->OBJ(x)'),
    (`SUB', `SUB_needs:speak_with_animals'),
    (`SUB', `SUB_abilities:magic_powder'),
    (`SUB', `SUB_abilities:lord_of_baghdad'),
    (`SUB_abilities:magic_powder', 
    `rule:OBJ(x)&&laugh(x)->not(know(x, magic_word))')
    ]
effects: {
    `edges_gone': [
        (`SUB', `SUB_race:human')
        ],
    `edges_appeared': [
        (`SUB_abilities:wealth', `SUB'),
        (`SUB_race:OBJ', `SUB')
        ]
    }
\end{verbatim}

There are similarities between all three occurences. They all share one of the
effects, namely that the race of the subject becomes the object
(\texttt{`SUB\_race:OBJ', `SUB')}.
This is a good property, as it facilitates this rule to be applied generally for all
transformations in future stories, even if the object that the person has been
transformed in has not been observed in another story.

Another common property of these rules is that the subject loses their own race
(this is denoted by \texttt{(`SUB', `SUB\_race:animal')} being one of the
effects of the transformation in the first two examples and 
\texttt{(`SUB', `SUB\_race:human')} being the effect in the third example). However, this similarity is
less pronounced than the previous one, since the race the character started out
with was not specified in the action. The first example was obtained from the
story event \texttt{`the frog transforms into a human.'}. If the story event had
consistently been in the form of \texttt{`the
frog transforms from an animal into a human.'}, then the resulting rule that
would have been learned would have been more general like the previous rule, 
since the race that the frog used to have (animal) would have its own variable,
making it more similar to a transformation in the opposite direction (for
example \texttt{the caliph transforms into an animal}).
%TODO try again, apparently

As said earlier in this section, the transformations in `Caliph Stork' were different from the one in `The Frog
Prince', since in `Caliph Stork' there was a magical powder with its associated
rules (the character using the powder had to speak a magical word to transform, and would forget the
word if they laughed in animal form), while in `The Frog Prince' there was no
explicit rule governing the magic. The rules of this magic powder can be seen in
the preconditions, since they were also represented as rule nodes connected to
the characters who used the powder to transform. For `The Frog Prince' there was no
specific rule, so there are no additional rules added to the precondition. In a
generation scenario, the program could choose between these examples, either
generating a scenario with complex magical rules (as in `Caliph Stork') or a 
scenario without these rules (like in `The Frog Prince').
This highlights that
though the knowledge base is able to recognize regularities in the data (like
the changing of the subject's race), it is also able to represent diversity in
its rule system (a generation program has access to diverse examples of
variations within the same rule).

Of course, the evaluation has been done on just two stories, since the
annotation for now is a time-intensive process. Because of this no actual story
generation has been done, but this thesis should be seen as more of a proof of
concept: patterns can be recognized, and if more annotated stories were
available (either through annotation or, more feasibly, through more
sophisticated natural language processing), the program could learn more interesting patterns. When it is
possible to automatically annotate the stories, it should be more feasible to
generate actually interesting stories, since the program will be able to learn a
larger amount and variety of rules.

\section{Conclusion and future work}

In this paper it is shown that the model can capture the preconditions and
implications of actions. Also, since the model keeps a list of probable actors
around for \texttt{obj}, \texttt{subj} and \texttt{dat},
it can capture how these character archetypes would act in a story. Because of the
way the model is queried however, these are not rules set in stone. A princess
might not murder often, but there is a small chance that it does happen. In
fact, a story where the princess would murder could be considered original, if
handled well (with a logical justifaction arising from a set of more probable
events).

As briefly alluded to in the Model section, the next step is to let a planning
algorithm generate stories based on the obtained rule probabilities. A possible
approach has been outlined in that section, but perhaps even more sophisticated
planning is possible, either mixing probable and improbable events on the fly,
or designing stories with more probable and improbable parts.

The current approach uses variables to keep track of how many times a certain
character was involved with an action and in what way. In a fairy tale context
this works, since characters are usually not named but are just known by their
archetype (the princess, the king, the witch, etc.). One could argue that this
is just a unique feature of fairy tales, but other genres have certain conventions
too. For example, in film noir, there are also certain stock characters (the
hard-boiled detective, the mysterious nightclub singer), but usually these are
known by their names, not by their archetype. An extension of this algorithm
could have characters with attributes, and let the probabilities be dependent on
the character attributes instead of the characters themselves.

Another way this approach could be extended is to get some more sophisticated
natural language parsing being done. The stories the current model was trained
on were manually annotated, but using only the information available from the
story synopses available from Wikipedia or similar sources. The challenges here
would be parsing the natural language found in the synopses, recognize the
entities, their properties and how they interact.
This would still be easier to accomplish than feeding the algorithm the complete
story as it might be found in a book, since then the algorithm would also have
to filter out which parts are actually rellevant for the plot.
Of course, the representation obtained automatically would not be as rich as a
humanly annotated one, but with sufficient stories it should be possible to fill
in the gaps, even without modifying the current way of probability computation
for rules.


\bibliographystyle{plain}
\bibliography{thesis.bib}

\end{document}
